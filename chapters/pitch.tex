\section{Sound properties}
A \textbf{sound} is just a bunch of air particles which vibrate really quickly into thin air. They do so on \emph{longitudinal} waves. The vibration in the air is produced by some sort of vibrating device (e.g. the membrane of a PC speaker) and captured by some other device (like the membranes into our ears). In the case of our ears, the brain processes the captured vibrations into sound perceptions.

The number of vibrations per second is expressed through the \textbf{frequency} $f$, expressed in \emph{Hertz}. Higher frequencies produce higher sounds. Conversely, if the signal has a bigger \textbf{amplitude} it is perceived as a louder sound. The timbre is instead related to the overall waveform shape and therefore to the Fourier representation of the signal.

Therefore, to summarise:
\begin{itemize}
    \item \textbf{Frequency} translates to \textbf{pitch}.
    \item \textbf{Amplitude} translates to \textbf{volume}.
    \item \textbf{Waveform shape} translates to \textbf{timbre}.
\end{itemize}

\section{Pitch}
Unfortunately, the relationship between frequency and pitch is nonlinear. In fact, it is \emph{exponential}.

Moving in octaves (in pitch space) to us feels like a linear increase, while in frequency the same movement is exponential (or also, the frequency increase is linear, but the pitch increase is logarithmic). As an example moving a sound to the next note involves \emph{doubling} its frequency.

Therefore, if we express as $s$ the multiplicative factor used to scale the frequency so as to obtain the corresponding pitch raised by a semitone:

$$s^{12} = 2 \Rightarrow s = 2^{\frac{1}{12}}$$

For translating a given pitch into a frequency we can use the following:

$$f = f_\textbf{C0} \cdot 2^{o+\frac{p}{12}}$$

Where $p$ is the number of semitones from the $C$ pitch class and $o$ is the number of octaves.

The reverse can be computed through logarithms and fixed / fractional part functions.

Nevertheless, from this we gather a pretty interesting result: \textbf{the pitch we perceive as a linear increase translates in reality as an exponential increase in frequency}.

This means that the most interesting octaves in music (C0 - C10) are effectively translated to a very narrow set of frequencies, up to 1000 Hz.

\subsection{Pitch class and octave equivalence}
The set of C pitches is called the \emph{C pitch class}. This can be generalized.

\begin{definition}[Pitch class]
    A pitch without its octave.
\end{definition}

We recognize pitches which share the same pitch class as different instances of the same note. In fact, these pitches are $n$ octaves apart so we should say that we recognize pitches \emph{which are $n$ octaves apart} as different instances of the same note.

Why is this so? This is because the waveforms of two notes from an octave interval perfectly line up together.

If we make an equivalence relation out of the notion of \emph{pitch space modulo octave equivalence} we are able to obtain the \textbf{pitch class space}.

\subsection{Temperament}
The modern western musical system is said to be a \textbf{12-tone equal-tempered} one.

\begin{definition}[12-tone equal-temperament]
    A musical system in which an octave is subdivided into a set of 12 equal semitones (equal = there is a constant amount between every semitone in pitch space).
\end{definition}

However, more than two centuries ago, composers actually used systems which were not equally-tempered. The semitones had different sizes across the octave, therefore leading to uneven intervals, be it for the better or for the worse. In fact, some of these intervals sound \emph{better} than their equally-tempered counterparts; however, these intervals lead to inconsistencies into other intervals / parts of the octave, which sound drastically worse than their equally-tempered counterparts (these intervals were called \textbf{wolf intervals}).